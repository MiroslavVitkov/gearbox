% Specification
% Gear Box for Machining Center
% P = 11kW
% n_output_max = 5000 min^{-1}

\documentclass{article}
\usepackage[nomessages]{fp}  % http://ctan.org/pkg/fp
\begin{document}

\FPeval\vExample{3.14*2.72}
\vExample

\subsection{Motor selection}
From [4] we select the Siemens 1PH209 spindle motor.
It is the recommended built-in motor for standard machine tool spindle.
Outside diameter is 205mm, output shaft diameter 67mm.
Nominal speed 1500rpm, maximum 10000rpm.

\subsection{Kinematic diagram (plan for rotational velocities)}
Let us suppose the goal of our gearbox is to operate the motor in its most efficient region.
Furthermore, let us select a low-speed gear of 1500rpm in addition to the required by specification 5000rpm gear.
Sonsequenctly, we need transfer ratios of 1.000 and 3.333.
%TODO: draw respective diagram

\subsection{Components sizing}
\subsubsection{Axles}
\subsubsection{Gears}
\subsubsection{Shifter}
\subsubsection{Casing and seals}
\subsubsection{Operating temperature}

\end{document}

% References
% 1. Machine Desing II, Module 2 - GEARS, Lecture 17 – DESIGN OF GEARBOX, Prof. K.Gopinath & Prof. M.M.Mayuram
% 2. Design Basics of Industrial Gear Boxes, Andrzej Maciejczyk, Zbigniew Zdziennicki, Technical University of Lodz
% 3. Example of gearbox calculation, www.mitcalc.com
% 4. https://www.industry.usa.siemens.com/drives/us/en/electric-motor/mc-motors/direct-drive-motors/Documents/mtr-direct-drive-brochure.pdf  (page 16)
