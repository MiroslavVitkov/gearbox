% Specification
% Gear Box for Machining Center
% P = 11kW
% n_output_max = 5000 min^{-1}

% Define a command to print variables as they get assigned.
% Use like \SetVar{variable_name}
% Command written by Heiko Oberdiek (http://tex.stackexchange.com/a/269392/45581) and edited by me.
\newcommand*{\SetVar}[2]{%
  \FPset{#1}{#2}
  \begingroup
    \fontencoding{T1}%
    \fontfamily{lmvtt}\selectfont % variable typewriter font
    % alternative: \ttfamily
    \detokenize{#1}% make _ to character
  \endgroup
  ~=~%
  \FPprint{#1}%
  \,%
  \ExtractUnit{#1}%
}
\newcommand*{\ExtractUnit}[1]{%
  \expandafter\ExtractUnitAux\detokenize{#1_}\relax
}
\begingroup
  \catcode`\_=12 %
  \gdef\ExtractUnitAux#1_#2\relax{%
    \ifx\\#2\\%
      #1
    \else
      \ExtractUnitAux#2\relax
    \fi
  }
\endgroup

% Command to print the value of a variable, together with the measurement unit.
% Written by me.
\newcommand*{\PrintVar}[1]
{
  \FPprint{#1}
}

\documentclass{article}
\usepackage[nomessages]{fp}  % http://ctan.org/pkg/fp
\begin{document}

\section{Requirements}
\SetVar{P_output_kW}{11}          \\
\SetVar{P_output_max_rpm}{5000}

%P_output_kW = \FPprint{P_output_kW} kW
%n_output_max_rpm = \FPprint{n_output_max_rpm} rpm

\section{Motor selection}
From [4] we select the Siemens 1PH209 spindle motor.
It is the recommended built-in motor for standard machine tool spindle.
Outside diameter is 205mm, output shaft diameter 67mm.
Nominal speed 1500rpm, maximum 10000rpm.

\section{Kinematic diagram} % (plan for rotational velocities)
The goal of the gearbox is to provide sufficient torque to the spindle at low motor speeds (below $n_{nominal}$).
For selecting the low gear we utilize the rule of thumb to choose a transfer ratio between $1/3$ and $1/4$:
$$ TR_{low} = 1/4 $$
For selecting the high gear, we try to match the maximum speed of the motor with the required maximum spindle speed by specification:
$$ TR_{high} = n_{spindle, max} / n_{motor, max} $$

Let us suppose the goal of our gearbox is to operate the motor in its most efficient region.
Furthermore, let us select a low-speed gear of 1500rpm in addition to the required by specification 5000rpm gear.
Sonsequenctly, we need transfer ratios of 1.000 and 3.333.
%TODO: draw respective diagram

\section{Components sizing}
\subsection{Gears}
Requirements: Design input shaft gears G11 and G12 and output shaft gears G21 and G22
Both center distances (G11-G21 and G12-G22) must be equal.
All gears must be capable of transmitting the rated torque \PrintVar{P_output_kW} plus efficiency losses.
Gear teeth must be shaped with a cutoff or round-off to assist gear engagement.  % зъбозаобляне
Conventional spur gears with involute teeth profile have been selected for this application.
\subsubsection{Contact ratio}
From [5] we know that Contact ratio is defined as
$\epsilon_a = 1.88 - 3.2 (1/z_1 + 1/z_2)$ % TODO: epproximately equal
and "for smooth and quiet operation, the contact ratio should not be less than 1.2".
\subsubsection{Standard Basic Rack Tooth Profile}
Due to the numerous advantages of using standard tooth geometry, we decide to adhare to the Standard Basic Rack Tooth Profile.
$$\alpha = 20$$  % pressure angle  TODO: degrees
$$h_a = 1.00m$$ % addendum
$$c = 0.25m$$ % bottom clearance
$$h_f = 1.25m$$ % daedendum
$$\rho = 0.38m$$ % fillet radius
$$h_w = 2.00m$$ % active tooth depth
$$h = 2.25m$$ % whole depth
$$s = (\pi m)/ 2$$ % tooth thickness
\subsubsection{Loading and forces}
$$F_t = 2T / d$$ % tangential force (T is torque)
$$F_r = F_t tan\alpha$$
$$F_n = F_t / cos \alpha $$ % resulting normal force
% TODO: tooth failure e.g. from page 202, but is not complete

\subsection{Couplings}
Requirements: Design a way to connect the Input shaft to the driving motor and the output shaft to the spindle.
We select a rigid coupling, because a flexible coupling could reduce the accuracy of the spindle rotation.
Furthermore, we narrow down to flange couplings for their superior vibration resistence and rigidity.
The stresses in the bolt bodies are:
$$\tau_{av} = (8 T_{max}) / (D \pi d^2) < \tau_{all}$$  % where Tmax - maximum transmitted torque = Tnom * K; K - service factor
% shear force on each bolt body: F = (2 T_{max}) / (z D)

\subsection{Shafts}
Requirements: desing Input shaft and Output shaft with appropriate key / spline / press  joints, capable of transmitting 11kW plus efficiency losses.
As the requirements match, we will callculate only a single shaft design.
\subsubsection{Static loading}
1. Determine length of shaft L. \\
2. Determine position of supports $a_1$ and $a_2$. \\
3. Determine shaft loads with their magnitude and point of application. \\
4. Select shaft material and its yield strength $\sigma_y$. \\
5. Select a safety factor S. \\
%TODO: draw shaft loading diagram
%etc p 158
\subsubsection{Fatigue loading}
\subsubsection{Shaft deflection}
\subsubsection{Shaft vibrations}
\FPeval\vExample{3.14*2.72}
\vExample

\subsection{Bearings}
Requirements: Select bearings of appropriate dimensions, precision, axial and radial stifness and sufficient vibration resistence.
Because sliding bearing cannot provide adequate wear resistence and heat rejection, we need rolling element bearing.
On the other hand, due to the complexity of designing a full-film lubrication system, we will rely on mixed film lubrication - the same used for the gear train.
We can expect a coefficient of friction between 0.04 and 0.1.
Because of the negligable axial load and the nessecity for high rigidity of the Output shaft, we select radial roller bearing type.

\subsection{Shifter}
\subsection{Casing and seals}
\subsection{Operating temperature}

\section{References}
1. Machine Desing II, Module 2 - GEARS, Lecture 17 – DESIGN OF GEARBOX, Prof. K.Gopinath \& Prof. M.M.Mayuram \\
2. Design Basics of Industrial Gear Boxes, Andrzej Maciejczyk, Zbigniew Zdziennicki, Technical University of Lodz \\
3. Example of gearbox calculation, www.mitcalc.com \\
4. https://www.industry.usa.siemens.com/drives/us/en/electric-motor/mc-motors/direct-drive-motors/Documents/mtr-direct-drive-brochure.pdf  (page 16) \\
5. Principles of mechanical engineering, Liubomir Dimitrov

\tableofcontents
\end{document}
